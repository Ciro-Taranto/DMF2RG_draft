 \documentclass[a4paper,11pt]{article}

\usepackage[T1]{fontenc}
\usepackage[english]{babel}
\usepackage{epsfig}
\usepackage{graphicx}
\usepackage{color}
\bibliographystyle{plain}

\begin{document}

\title{Status report about charge instability}

\maketitle 

%\newcommand{\MPIStutt}{\affiliation{Max-Planck-Institute for Solid State Research, 70569 Stuttgart, Germany}}

%\author{Demetrio Vilardi }   \MPIStutt 

%\author{Ciro Taranto} 		 \MPIStutt

%\author{Walter Metzner }     \MPIStutt 



\begin{abstract}

We collect some results obtained by means of different implementations of fRG equations using a full frequency dependent vertex. 
It emerges a very peaked structure in the charge-channel for finite frequency-transfer, that in some region of the parameter space becomes divergent. Such a divergence has no obvious physical interpretation.  
The peaked structure seems to be characteristic of the frequency dependence of the vertex, as it is shown by means of simpled diagrams. 
On the other hand the \emph{divergence} of this structure may be very sensitive to the detailed structure of the Green's function used in the calculation, i.e., very sensitive to the use, or not, of dressed propagators, even when the correction to the self-energy appear to be small (i.e., self-energy Fermi liquid-like).  

\end{abstract}

\maketitle

\section{fRG without self-energy} 				
\label{sec:frgnoself}
  The results shown in this section are obtained in standard fRG using an interaction cutoff: $G_0^\Lambda = \Lambda G_0$. The calculations are performed on the Matsubara frequency axis for temperature $T=0,08 t$, where $t$ is the nearest neighbors hopping.  
The momentum dependence of the vertex is treated by means of a form factor decomposition, while keeping 29 patches in the respective bosonic momentum transfer . The critical scale is fixed by the condition that the absolute value of one of the channels exceeds a value of 1200$t$ (similar results when analyzing the suscpetibility). 
 

\begin{equation}
\label{sdmft}
a = a ;  
\end{equation}


\begin{figure}
\includegraphics[scale=0.4]{vanHove_scan_critical_lambda_phi.eps}
\caption{Critical scale in full frequency fRG (interaction cutoff) as a function of the nearest neighbors hopping and for van Hove filling. The color of the symbol indicates the kind of instability that is realized.  } \label{dictionary}

\end{figure}

\begin{figure}
\includegraphics[scale=0.4]{vanHove_plus_scan_critical_lambda_phi.eps}
\caption{Critical scale in full frequency fRG (interaction cutoff) as a function of the nearest neighbors hopping and for van Hove filling + 7 \% . The color of the symbol indicates the kind of instability that is realized.} 
\end{figure}

%\section{DMF$^2$RG without self-energy}	
%\label{sec:dmf2rgnoself}
  %In this section we present results obtained with DMF$^2$RG.  We use an interpolative cutoff: $[G_0^\Lambda(\mathbf{k},\omega)]^{-1} = (1-\Lambda)*[G_0(\mathbf{k},\omega)]^{-1}+\Lambda \mathcal{G}_{\mathrm{AIM}}(\omega)$, ($\mathcal{G}_{\mathrm{AIM}}$ is the propagator of the DMFT self-consistent Anderson impurity model associated with the lattice under consideration. The initial condition for the vertex and the self-energy is per se frequency dependent. 
Apart for this, the implementation is equivalent to the one used in fRG, i.e., same frequency and momentum treatments. 
The self-energy is however better-behaved, compared to the fRG situation: we speculate that this is a consequence of the fact that the flow only has to compute the deviation from the DMFT self-energy. Hence the results with the self energy feedback of this section are numerically more stable.  

The results presented here are (in the usual fRG units) for $U=6.8t$, $T=-0.08t$, filling $n=0.85$, and next neighbors hopping $t'=-0.03$.  


\begin{figure}
\includegraphics[scale=0.5]{vanHove_scan_critical_lambda_phi.eps}
\caption{Diagrammatic elements used in the text. The Yukawa coupling already refers to the case of spin boson. For a density boson there is a plus sign. } \label{dictionary}

\end{figure}


%\section{DMF$^2$RG with self-energy}	
%\label{sec:dmf2rgself}
  %\input{content/dmf2rgself.tex}

%\section{Perpendicular ladders}				
%\label{sec:perpendciular}
  %As a poor man way to understand the frequency dependence of the charge channel we introduce a "\emph{perpendicular ladder}" approximation, tailor to study the influence of ladder-type diagrams in the particle hole crossed channel to the charge channel. 

\begin{itemize}
\item We assume that the magnetic interaction (for some momenta) is leading and therefore, we compute it by RPA, neglecting feedback from other channels (overestimating) of the magnetic fluctuations. 

\item From this we compute an effective interaction for the charge channels, by appropriate frequency and momentum translation ; 

\item Magnetic RPA local effective interaction:

\item \textbf{SCRIVERE FORMULA DELLA BOLLA!}

\begin{equation}
  U_{eff}(\Omega) =\int_{\boldsymbol{Q}} \frac{U}{ 1 - U \Pi^{\Omega}_{\boldsymbol{Q}} }
\end{equation}
(only the momentum integrated interaction enters in the momentum decomposition considered)

\item the charge channel is computed by RPA: 
\begin{equation}
  \Phi_{charge}^{\Omega,\nu,\nu'}(\boldsymbol{Q}) = U_{eff}(\Omega) 
  \left[ 1+  \Pi^{\Omega,\nu}_{\boldsymbol{Q}} U_{eff}(\nu'-\nu) \right]_{\nu,\nu'}^{-1},
\end{equation}
where it is very important to understand the equations as matrix (in the frequency space) qualities. 

\item As one can see in Fig. \ref{Perpladder} the frequency structure obtained for the charge channel reflects the one of fRG (and also of DMF$^2$RG). 
This formula predicts the frequency structure, while certainly giving an overestimate for the effective value of the channel, which has not  to be taken quantitatively. 

\item From figure \ref{bubble} one can understand both the importance of the frequency dependence of the interaction and why the structure is particularly large for the first non vanishing bosonic frequency. 
Indeed: if the interaction was featureless in the frequency, only the frequency summed bubble would enter in the equations. This, in turn vanishes for $\mathbf{Q}=(0,0)$ and finite frequency.  

\end{itemize}

\begin{figure}
\includegraphics[scale=0.25]{images/Perp_ladder_density.png}
\caption{Charge channel calculated with RPA method for $\Omega=0$ (left) and $\Omega=2\pi/\beta$ (right). }
 \label{Perpladder}
\end{figure}

\begin{figure}
\includegraphics[scale=0.7]{images/Bubble_ph.png}
\caption{PH Bubble as a function of fermionic frequency $\nu$ at $\boldsymbol{Q}=(0,0)$. }
 \label{bubble}
\end{figure}

%\section{Self-energy in fRG}				
%\label{sec:frgself}
  %\begin{itemize}

\item The self-energy flow is unreliable. It suffers from the finite frequency box of the vertex. 

\item Self energy inclusion suppresses the charge channel in favor of FM fluctuations, see Fig. \ref{selfsuppressfrg}. 

\item Self energy lowers the critical scale.

\item For the set of parameters considered ($U=4t$, $T=0.08t$, $t'=-0.10$, van Hove filling) the self energy seemed to be Fermi liquid like and did not show any sign of divergent behavior. 

\item In figure \ref{fermisurface} one can see that the self energy affects the Fermi surface which gets broadened, as we deduce from the momentum dependent occupation shown. 

\end{itemize}

\begin{figure}
\includegraphics[scale=0.7]{images/sevsnose.png}
\caption{Max values of the channels as functions of cutoff scale, with and without self energy. }
\label{selfsuppressfrg}
 \end{figure}
 
 \begin{figure}
 \includegraphics[scale=0.3]{images/Fermi_occupation_sevsnose.png}
 \caption{Occupation in the BZ: left non interacting, right with self energy from fRG.}
\label{fermisurface}
 \end{figure}


  \bibliography{refs}

\end{document}
